%!TEX root = ../main.tex
\section{Συμπεράσματα}

Η όλη διαδικασία για την ανάπτυξη και υλοποίηση ενός συστήματος υπολογιστικής όρασης σε FPGA με τη μέθοδο HLS είναι αρκετά πολύπλοκη αλλά αποδοτική για πολλές περιπτώσεις εφαρμογών. Εμφανίζει σημαντικά πλεονεκτήματα αλλά και μειονεκτήματα που πρέπει να συνεκτιμηθούν πριν από την έναρξη οποιουδήποτε project.

Η ανάπτυξη του αλγορίθμου για την εφαρμογή μας γίνεται αρχικά στον ΗΥ σε κάποια γλώσσα προγραμματισμού υψηλού επιπέδου και έτσι δεν απαιτείται προσπάθεια εκμάθησης μία νέας γλώσσας (VHDL, Verilog). Γνώσεις κάποιας γλώσσας περιγραφής υλικού μπορούν να διευκολύνουν σε ορισμένες περιπτώσεις την υλοποίηση της εφαρμογής μας. Η μεταφορά, επίσης, του αλγορίθμου μας στο Vivado HLS είναι απλή διαδικασία.Επίσης, η αποσφαλμάτωση, η αλλαγή κάποιων βημάτων και η βελτιστοποίησή του είναι σχετικά απλά βήματα και μπορούν να πραγματοποιηθούν σε μικρό χρονικό διάστημα. Ως παράδειγμα, αναφέρουμε ότι μετά την ανάπτυξη του κώδικα μας, επετεύχθη χωρίς μεγάλο κόπο η μείωση των απαιτούμενων κύκλων από $3.1\times 10^6$ σε $9.4\times 10^4$.

Πρέπει όμως σε αυτό το σημείο να τονιστεί ότι δεν είναι όλοι οι κώδικες συνθέσιμοι και πρέπει να ληφθεί ειδική μέριμνα κατά τη συγγραφή ώστε να τηρηθούν οι περιορισμοί του HLS.
Επίσης, σημαντικό ρόλο στην απόδοση του αλγορίθμου παίζει και ο τρόπος αντιμετώπισης του προβλήματος. Αρκετές φορές, ο συμβατικός τρόπος γραφής κώδικα δεν "ταιριάζει" στο FGPA με αποτέλεσμα τη μειωμένη απόδοση του συστήματος.

Σχετικά, με τη διαδικασία ελέγχου της ορθής λειτουργίας του αλγορίθμου, αυτή πραγματοποιείται αρκετά γρήγορα και εύκολα καθώς το testbench αναπτύσσεται σε γλώσσα προγραμματισμού υψηλού επιπέδου. Από τους ελέγχους που πραγματοποιήθηκαν διαπιστώθηκε η πλήρης ταύτιση των εικόνων που επεξεργάστηκαν στο FPGA με αυτές που παράχθηκαν στον ΗΥ.

Είναι γεγoνός ότι η αρχική ανάπτυξη του πλήρους συστήματος απαιτεί άριστη γνώση των επιμέρους τμημάτων του. Συνεπώς, είναι αναγκαία η μελέτη μεγάλου όγκου πληροφοριών από τα εγχειρίδια τόσο του Vivado HLS όσο και του χρησιμοποιούμενου εξοπλισμού της Xilinx. Στη συνέχεια βέβαια τα πράγματα είναι ευκολότερα όταν έχει αποκτηθεί η σχετική εμπειρία.

Παρά το γεγονός όμως ότι η διαδικασία ανάπτυξης του αλγορίθμου και η μεταφορά του στο HLS είναι σημαντικά συντομότερη από την ανάπτυξη του σε HDL, πολλές φορές εμφανίζει σφάλματα. Όσο περισσότερα βήματα γίνονται τόσο μεγαλύτερη είναι και η πιθανότητα εμφάνισης σφαλμάτων.

Καταληκτικά, συμπεραίνουμε ότι είναι δυνατή η υλοποίηση ενός συστήματος CV σε ένα εύλογο χρονικό διάστημα, επιτυγχάνοντας ικανοποιητικά επίπεδα απόδοσης. Το σύστημα που αναπτύχθηκε είναι εύκολο να τροποποιηθεί μέσω απλών διαδικασιών σε σύντομο χρόνο. Αυτή η μέθοδος ανάπτυξης υλικού παρά το γεγονός ότι ειναι σχετικά καινούρια μπορεί να χρησιμοποιηθεί στη βιομηχανία για ανάπτυξη πρωτοτύπων και συστημάτων που δεν απαιτούν ακρίβεια σε επίπεδο bit. Απαιτείται όμως σημαντική εμπειρία στη συγγραφή κώδικα με κατάλληλο τρόπο ώστε να αναπτυχθεί η επιθυμητή αρχιτεκτονική υλικού. Παρ' όλα αυτά, είναι φανερό ότι το HLS είναι ένα συνεχώς βελτιούμενο εργαλείο το οποίο αξίζει να βρίσκεται στο πεδίο ενδιαφέροντος των σχεδιαστών υλικού.

\section{Μελλοντικές εργασίες \& βελτιώσεις}

Κλείνοντας, θα πρέπει να αναφέρουμε ότι τα αποτελέσματα της παρούσας εργασίας μπορούν να συμπληρωθούν μελλοντικά με περαιτέρω βελτιώσεις ή προσθήκη νέων δυνατοτήτων.
\begin{enumerate}
	\item Τροποποίηση του αλγορίθμου για την υποστήριξη εικόνων μεγαλύτερης ανάλυσης καθώς και περαιτέρω αύξηση της απόδοσης του. Αυτό μπορεί να επιτευχθεί με την υλοποίηση παραθύρων για την επεξεργασία των νέων δεδομένων ώστε να επιτευχθεί μείωση στο συνολικό χρόνο καθυστέρησης παραγωγής νέου δείγματος και εξοικονόμηση μνήμης στο FPGA. Μία ακόμη τεχνική είναι η διαίρεση της εικόνας σε $N$ μικρότερες και η διαδοχική επεξεργασία τους.
	\item Αύξηση της λειτουργικότητάς του αλγορίθμου μας με τη σύνδεση περιφερειακών για τη λήψη και αποστολή δεδομένων, όπως για παράδειγμα μέσω κάμερας ή διεπαφής HDMI.
	\item Αξιοποίηση των βιβλιοθηκών που παρέχονται από τη Xilinx όπως για παράδειγμα, η βιβλιοθήκη openCV \footnote{https://opencv.org/} η οποία παρέχει ολόκληρες λειτουργίας υπολογιστικής όρασης με τη μορφή απλών συναρτήσεων.
\end{enumerate}