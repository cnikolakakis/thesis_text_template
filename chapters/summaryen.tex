%!TEX root = ../main.tex
\pagestyle{plain}

\begin{center}
{\LARGE Abstract}\\[1cm]
\end{center}

\hyphenation{Vi-va-do}
\hyphenation{Pro-gram-ma-ble}

High computational applications, like algorithms of Computer Vision, require a lot of performance that cannot be provided by low cost personal computers. Such applications, due to their high complexity and the size of the data they process, put a significant strain on the computing infrastracture executing them. Therefore, they are mainly implemented on application-specific integrated circuits (ASIC) or Field Programmable Gate Array Circuits (FPGA).

Their implementation requires a large amount of highly trained engineers and large amount of time. However, both the cost and the duration of their implementation must be minimized.

In this diploma thesis, we focus on providing a hardware/software design and an embedded system that implements a resource hungry Computer Vision algorithm: Canny Edge Detector. Comparing to the usual way of dealing with the implementation of such algorithms we chose to go with Xilinx's Vivado High-Level Synthesis (HLS). After developing the algorithm in a high-level programming language, for example C/C++, HLS synthesizes it and produces a RTL hardware design.

Taking advantage of HLS' optimizations and Xilinx's Vivado platform we observe that the development, debugging, and implementation times can be significantly reducded. The development board used in this thesis is the ZC702 that features a Zynq-7000 All Programmable SoC produced by Xilinx.\\

\noindent
\textbf{Keywords}: FPGA, High-level Synthesis, Vivado HLS, Canny Edge Detector, Zynq-7000 All Programmable SoC, Edge Detection, Vivado HLS, Vivado, ARM