%!TEX root = ../main.tex
\pagestyle{plain}

\begin{center}
{\LARGE Περίληψη}\\[1cm]
\end{center}

Εφαρμογές με υψηλές υπολογιστικές απαιτήσεις, όπως για παράδειγμα αλγόριθμοι μηχανικής όρασης, απαιτούν επιδόσεις που δεν μπορούν να ικανοποιηθούν από ηλεκτρονικούς υπολογιστές χαμηλού κόστους καθώς απαιτούν σημαντικό υπολογιστικό φόρτο, εξαιτίας της αυξημένης πολυπλοκότητάς τους αλλά και του όγκου των δεδομένων που χρησιμοποιούν. Για το λόγο αυτό η υλοποίησή τους πραγματοποιείται σε υλικό ειδικού σκοπού, όπως για παράδειγμα σε application-specific integrated circuits (ASIC)s ή se Field Programmable Gate Array circuits (FPGA).

Η υλοποίηση σε εξειδικευμένο υλικό απαιτεί την απασχόληση πολυπληθούς και ιδιαίτερα εξειδικευμένου προσωπικού καθώς και μεγάλο χρόνο ανάπτυξης. Για το λόγο αυτό είναι απαραίτητο οι διαδικασίες υλοποίησής τους αφενός να συντομευτούν πολύ, αφετέρου να μειωθεί το κόστος ανάπτυξής τους.

Σκοπός της παρούσας διπλωματικής εργασίας είναι η ανάπτυξη ενός συνδυασμένου συστήματος hardware/software και η υλοποίηση ενός ενσωματωμένου συστήματος για την υλοποίηση ενός τέτοιου απαιτητικού αλγορίθμου, συγκεκριμένα του Canny Edge Detector. Σε αντίθεση με το συνηθισμένο τρόπο αντιμετώπισης σχεδιασμού σε FPGA, θα αναπτύξουμε τον αλγόριθμο με τη βοήθεια του εργαλείου Vivado High-level Synthesis (HLS) της Xilinx, το οποίο μετασχηματίζει μία περιγραφή από γλώσσα γενικού σκοπού, υψηλού επιπέδου όπως η C σε μία περιγραφή υλικού (RTL).

\hyphenation{Vi-va-do}

Εκμεταλλευόμενοι τις δυνατότητες βελτιστοποίησης του Vivado HLS και γενικότερα της πλατφόρμας Vivado, μπορούμε να ελαχιστοποιήσουμε κατά πολύ το χρόνο ανάπτυξης και αποσφαλμάτωσης του αλγορίθμου μας και της υλοποίησης του συστήματος. Η συσκευή που χρησιμοποιήθηκε είναι το σύστημα ανάπτυξης της εταιρείας Xilinx, ZC702 με τον Zynq-7000 All Programmable SoC.\\

\noindent
\textbf{Λέξεις κλειδιά}: FPGA, Σύνθεση υψηλού επιπέδου, Canny Edge Detector, Zynq-7000 All Programmable SoC, Ανίχνευση ακμών,  Vivado HLS, Vivado, ARM